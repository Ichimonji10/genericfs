%%%%%%%%%%%%%%%%%%%%%%%%%%%%%%%%%%%%%%%%%%%%%%%%%%%%%%%%%%%%%%%%%%%%%%%%%%%
% FILE    : Kernel-Development.tex
% AUTHOR  : (C) Copyright 2012 by Peter C. Chapin
% SUBJECT : Configuring Linux for kernel development..
%
% This document describes how to set up a Linux kernel for kernel study and development. It
% covers how to compile the kernel and how to set up and use various kernel debugging and code
% browsing tools.
%
%       Peter C. Chapin
%       Computer Information Systems
%       Vermont Technical College
%       Randolph Center, VT. 05061
%       PChapin@vtc.vsc.edu
%%%%%%%%%%%%%%%%%%%%%%%%%%%%%%%%%%%%%%%%%%%%%%%%%%%%%%%%%%%%%%%%%%%%%%%%%%%

% ++++++++++++++++++++++++++++++++
% Preamble and global declarations
% ++++++++++++++++++++++++++++++++
\documentclass{article}

\usepackage{hyperref}

% \pagestyle{headings}
\setlength{\parindent}{0em}
\setlength{\parskip}{1.75ex plus0.5ex minus0.5ex}

% ------------
% New Commands
% ------------

% Add commands in alphabetical order.
\newcommand{\command}[1]{\texttt{#1}}    % For formatting commands.
\newcommand{\filename}[1]{\texttt{#1}}   % For formatting file names.
\newcommand{\todo}[1]{\textit{TODO: #1}}

% ----------------
% New Environments
% ----------------

% An environment to display a sequence of commands.
\newenvironment{commands}
  {\begin{quote} \tt}
  {\end{quote}}


% +++++++++++++++++++
% The document itself
% +++++++++++++++++++
\begin{document}

% ----------------------
% Title page information
% ----------------------
\title{Linux Kernel Development}
\author{\copyright\ Copyright 2012 by Peter C. Chapin}
\date{Last Revised: December 15, 2012}
\maketitle

% -----------------
% Table of contents
% -----------------
\pagenumbering{roman}
\tableofcontents
\newpage
\pagenumbering{arabic}

\section{Introduction}

This document describes how to set up a Linux system for kernel study and development. It
contains specific instructions for compiling an experimental kernel and for configuring and
using various kernel debugging and code browsing tools. This document was originally prepared to
support the operating systems class at Vermont Technical College and some of the information it
contains targets that audience. However, much of the contents of this document would be useful
to anyone interested in studying the Linux kernel.

\section{Compiling the Kernel}

This section describes how to compile the Linux kernel. It is intended to support individuals
compiling the kernel for the first time. Note that this section assumes you are using a 3.x
version of the kernel. The instructions for compiling older versions may be different.

\subsection{Downloading and Unpacking}

Before compiling you will need to obtain a copy of its source code. Your Linux distribution
should come with the source code of the kernel (it's required by the GPL). However, because the
kernel source is large and because most users do not require it, the kernel source is not
normally installed by default. Furthermore, each distribution tends to specialize the kernel in
various ways. To build a new kernel from a particular distribution's kernel source package, you
should consult the documentation for that distribution.

If you are interested in kernel development or if you want to always use the latest kernel, I
recommend that you download the "stock" kernel source from http://www.kernel.org/. This site is
the official distribution point for Linux kernels. The kernels on this site are generic in the
sense that they haven't been customized for any particular Linux distribution. As a result you
may have to configure the kernel (as described below) in a non-default way before it will boot
your system cleanly.

This document describes the build process specifically for kernel version 3.6.11 on 64~bit
Ubuntu Linux 12.04 (precise). The commands below reflect this version number and platform. If
you are using a different version or building on a different platform, modify the commands as
appropriate. Note that you may need to install some extra packages on your system in order to do
kernel development. This document does not describe which packages are necessary nor how to
install them. Consult the documentation for your distribution for more information.

The kernel source is kept in a compressed archive called a ``tarball.'' For version 3.6.11, the
name of this archive is \filename{linux-3.6.11.tar.bz2}. The last component of the version
number is a release level. As 3.6 matures, it advances through several releases.

Once you download the tarball you will need to unpack it.

\begin{commands}
\$ bunzip2 < linux-3.6.11.tar.bz2 | tar xf -
\end{commands}

Unpacking the tarball will create a \filename{linux-3.6.11} directory beneath the current
directory. While doing kernel development it is probably best to unpack the kernel somewhere in
your home directory. That will make it easy for you to do development under your normal user
account. You will need to be root to install the kernel, but you can build it as an ordinary
user.

\subsection{The C Library }

The application interface to the kernel is by way of the C library. In theory when the kernel is
updated, the C library needs to be rebuilt so that it can take into account any changes in
kernel-specific data structures provided by the new kernel. Applications that link to the C
library statically (that is \emph{not} using the dynamic shared library) would then also need to
be recompiled.

In addition some of the header files from the kernel source are usually also in
\filename{/usr/include} where applications that need them (for example applications making
direct system calls) can access them. Again, in theory, when the kernel is updated those header
files also need to be updated as well.

A Linux distribution installs a version of the C library and kernel headers that correspond to
the installed kernel. When you update your kernel using the distribution's normal update system,
these things are also updated if necessary. When you manually update your kernel, however, you
might have to update these things yourself.

That said, changes in the kernel that necessitate rebuilding the C library are relatively rare.
If the kernel you are installing is not too different than the one you are already using you can
probably get away without these additional complications.

In any case you might look at the contents of \filename{/usr/include/linux} and
\filename{/usr/include/asm*} (that is, all directories with names starting with `asm'). These
are the kernel headers installed by your distribution and used by the currently installed C
library. In contrast the corresponding directories under \filename{linux-3.6.11/include} contain
the headers for the kernel you just downloaded.

Some distributions use symbolic links from \filename{/usr/include} into the kernel source tree.
In that case, they will install the kernel headers under \filename{/usr/src} for the installed
kernel even if they do not install the full kernel source. In a case like this you can change
the symbolic links in \filename{/usr/include} to point at the new headers; however you should in
theory also rebuild the C library and statically linked applications if you do this.

If this sounds complicated and unreliable you aren't the only person who thinks that. There have
been discussions among the kernel developers to "fix" this situation, but I'm not sure how those
discussions have concluded.

\subsection{Compiling}

\subsubsection{Configure the Kernel}
\label{sec:configuring}

Before you compile the kernel you should ``configure'' it. This involves selecting which
features you want enabled in your kernel and which features should be compiled as modules that
can be loaded later. The configuration process creates two files: \filename{.config} in the root
of the kernel source tree and \filename{autoconf.h} in \filename{include/generated}\footnote{The
  \filename{autoconf.h} file is created as part of the build process. It does not exist until
  you actually compile the kernel for the first time.}. The \filename{.config} file gives the
make utility access to your desired configuration. Make uses this information to control which
files are compiled and how. The \filename{autoconf.h} file is included into the various kernel
source files (and also in external modules) and gives the C compiler access to your desired
configuration. Programmers can use \#ifdef/\#endif directives in the C source to selectively
compile different code depending on the configuration options chosen.

To create a configuration based on the currently running kernel, you will need to first obtain a
copy of the configuration file for the that kernel. Depending on how the running kernel was
configured a compressed version of that file might be in \filename{/proc/config.gz}. My Ubuntu
12.04 distribution stores a copy of the current configuration in
\filename{/boot/config-3.2.0-34-generic} where 34 is the current (as of the time of this
writing) release number of the Ubuntu kernel. You can read the file \filename{/proc/version} to
find the exact version of the running kernel. That information might be useful for determining
which configuration file is the most appropriate if there is more than one.

Copy the existing configuration file to the root of your new kernel source tree under the name
\filename{.config}. Then run the \command{make oldconfig} command to update that configuration
for use with the new kernel. For example:
\begin{commands}
\$ cp /boot/config-3.2.0-34-generic .config \\
\$ make oldconfig
\end{commands}

Starting with an existing configuration is generally desirable if you are building a kernel that
you intend to run on the same machine where you are doing the build. This allows you to
configure your new kernel to match the existing configuration as closely as possible resulting
in a minimum of problems with the new kernel.

There are two kinds of issues that will be reported by the \command{make oldconfig} command.
First, configuration options in the existing kernel that are not in the new kernel (because they
have been removed or renamed, or because the existing kernel has been extended by the distro)
produce warnings when they can't be mapped into the new configuration. Second, configuration
options in the new kernel that are not in the existing kernel (because they are new) will prompt
you to choose between `Y' (meaning compile the feature into the kernel), `N' (meaning don't
compile the feature), or in some cases `M' (meaning compile the feature as a module). You can
just hit `enter' to accept the default in most cases\footnote{You can also select `?' to get
  help information about an option.}.

After the initial configuration process is complete you can use
\begin{commands}
\$ make menuconfig
\end{commands}
OR
\begin{commands}
\$ make gconfig
\end{commands}

to further refine the configuration. You can also use these commands to change the configuration
later if desired. Note that \command{make menuconfig} requires that you have the curses terminal
handling library available and \command{make gconfig} requires that you be running the X Window
system with the GTK+ graphical libraries available.

If you are building a kernel for development and experimentation purposes you may want to enable
some debugging features in the kernel configuration. I invite you to explore the options under
the ``Kernel hacking'' heading. Even if you decide to not activate any debugging options at this
time it would be good for you to be aware of the possibilities in case you want to try them
later.

Note that debugging options and the checks they imply will impact the performance of your kernel
in terms of both space and time. This is why many of them are off by default. In fact, some
checks are so expensive that I do not recommend using them in a kernel built for general use.
Consult the help information on each option for more information.

If you wish to debug your kernel using a source level debugger you will want ``Compile the
kernel with debug info'' turned on also (this is the default in the Ubuntu 12.04 configuration).
Be aware that this option greatly increases the amount of disk space required to build the
kernel since every object file produced by the compiler contains debugging symbols. This option
is appropriate if you are creating a User Mode Linux kernel (see Section~\ref{sec:UML}) or if
you plan to use a kernel debugger or a tool to analyze kernel crash dumps.

\subsubsection{Doing the Build}

To actually build the kernel and all of its supporting modules do
\begin{commands}
\$ make
\end{commands}

A kernel build takes some time. There is a lot of code. Note that this will compile most drivers
(as modules) even though your system will never use them.

\subsubsection{Copying Kernel Files}

You do not need to be root to configure and compile the kernel. If you unpack the kernel source
in an area where you have read/write permission, you should be able to build it as an ordinary
user. However, you do need to be root to copy the new kernel to a place where it can be used to
boot the machine.

Once the kernel has been built you should also copy the various modules that were built to the
proper directory under \filename{/lib/modules} so that the running kernel can find them. This is
accomplished by doing
\begin{commands}
\# make modules\_install
\end{commands}

It is safe to do this even if you usually run a different kernel version. Each version has its
own private directory under the \filename{/lib/modules} directory.

You should next copy and rename three files from your freshly built kernel to the
\filename{/boot} directory. For example
\begin{commands}
\# cp -p arch/x86/boot/bzImage /boot/vmlinuz-3.6.11 \\
\# cp -p .config /boot/config-3.6.11 \\
\# cp -p System.map /boot/System.map-3.6.11
\end{commands}

The \filename{System.map} file contains a list of all symbols in the kernel and their
corresponding addresses. This can be useful for debugging and for interpreting stack traces in
kernel oops messages. \todo{why is it important for this file to be in \filename{/boot}?}

The \command{-p} option in the commands above causes \command{cp} to preserve the timestamps on
the copy. This means the times on the files in \filename{/boot} will reflect when those files
were orginally created. It is sometimes useful to retain this information for reference. However
the \command{-p} option will also preserve the file owner and permission settings. Thus after
copying the files you might want to manually change the owner and permissions on the new files
to match those already in \filename{/boot}. For example the configuration and system map files
are typically readable by everyone but the kernel image itself is typically only readable by
root.

\subsubsection{Making \filename{initrd}}

Because modern Linux systems are so highly modularized it is possible that the kernel will need
to load a module in order to read the file system. This creates a problem: how can it load a
file system support module from the file system? To get around this, Linux boots in two phases.
During the first phase the boot loader loads a pre-defined RAM disk image into memory along with
the kernel. The kernel uses this RAM disk image as it's initial root file system. Certain
programs and kernel modules can be loaded out of this RAM disk image. Once that is done, the
root file system is changed to be the normal hard disk and the usual start-up scripts are
executed.

Manually creating this initial RAM disk is a somewhat involved procedure. Fortunately there is
utility program named \command{mkinitramfs} that does most of the work. On a Ubuntu system the
command is
\begin{commands}
\# mkinitramfs -o /boot/initrd.img-3.6.11 3.6.11
\end{commands}

This creates a RAM disk using the same modules as in the existing configuration, except that it
will use the modules for the right kernel version. If you attempt to use the old RAM disk, it
will contain modules for the old kernel which won't load into the new kernel.

\subsubsection{Configuring GRUB}

Once you have \filename{vmlinuz-3.6.11} and \filename{initrd.img-3.6.11} in the
\filename{/boot} directory you only need to update your boot loader to provide an option to boot
the new kernel. The easiest way to do this is to clone the information for the existing kernel
to a new menu entry and modify the names of the kernel image file and RAM disk file. The precise
steps for doing this will depend on the boot loader you use.

On a Ubuntu system this is easily accomplished by using the \command{update-grub} command. This
command searches /boot for kernels installed there and composes a suitable GRUB menu for them.

The next time you boot your system if you press the left-hand shift key early in the boot
process you will see the GRUB boot menu and from there you can select your new kernel. \todo{Say
  more about setting up GRUB options.}

\subsection{Installing on Floppy}

\textit{This section is old needs to be rewritten. For one thing it needs to explain how to
  handle the initial RAM disk. For a second thing it should probably really talk about setting
  up a flash drive instead of a floppy (who has floppy drives?).}

The following instructions pertain to users who are booting Linux from a floppy disk. Note that
this is not the normal configuration (although it is sometimes useful in lab situations).

\begin{enumerate}

\item Make a copy of your boot floppy. Never overwrite a known working kernel with one that you
  just compiled! First make a backup of the working kernel and be sure that you can boot the
  working kernel if necessary.

  On Windows you can backup your boot floppy with the \command{diskcopy} command. Open a Windows
  command prompt and do

\begin{commands}
C:$\backslash$> diskcopy a: a:
\end{commands}

You will be prompted for the source disk and then for the target disk. Note that you can't just
copy the files from one disk to another! A boot floppy contains special information in the boot
sector that will not be copied by the usual file copying operations.

On Linux you can use the \command{dd} command to copy disks raw. Insert the source floppy and do
\begin{commands}
\# dd if=/dev/fd0 of=/tmp/floppy.img bs=1024
\end{commands}

Then insert the target floppy and do
\begin{commands}
\# dd if=/tmp/floppy.img of=/dev/fd0 bs=1024
\end{commands}

You can remove the temporary file afterward if you wish. See the manual page for the
\command{dd} command for more information.

\item Next insert the boot floppy where you want the new kernel to go and mount it. This can be
  done with a command such as
\begin{commands}
\# mount /dev/fd0 /mnt/floppy
\end{commands}

Use whatever directory is most appropriate if \filename{/mnt/floppy} is not available (there
should be a \filename{/mnt} directory at least).

\item Copy \filename{/usr/src/linux-3.6.11/arch/x86/boot/bzImage} to \filename{/mnt/floppy},
  renaming it to \filename{vmlinuz} in the process. This will overwrite the \filename{vmlinuz}
  on the floppy with the new kernel.

  There are some control files on the floppy as well that you could edit. However, if you use
  the same name (and you might as well since the floppy isn't big enough to hold both kernel
  images) the control files should already be configured properly.

\item \emph{Very Important!} Unmount the floppy before physically removing it. This is necessary
  because Linux keeps disk blocks in its cache even for floppy disks. This means that the entire
  file isn't actually put on the floppy until you unmount it.
\begin{commands}
\# umount /mnt/floppy
\end{commands}

\end{enumerate}

Now you can reboot the machine from your new boot floppy to check your new kernel.

\section{User Mode Linux}
\label{sec:UML}

Setting up a completely independent Linux system is a nice way to do Kernel development.
However, there are times when it may not be desirable. For example, if you want to do Kernel
development on a machine that you depend on for your normal work you should not risk running
experimental or debugging kernels directly on that system. One option is to use virtualization
software such as VMware or VirtualBox (to name just two of several available products). If your
primary machine is also running Linux, another approach that doesn't require any additional
software is User Mode Linux (UML).

The Linux kernel is cross platform and with suitable cross compilers can be compiled on one
platform for execution on another. User Mode Linux is treated as a special ``platform.''
However, the UML kernel runs on top of a host Linux system as an ordinary process. All access to
hardware is translated into system calls into the hosting system. UML thus allows you to run a
custom Linux kernel alongside your regular applications. You don't even need to be root on the
host system.

Another advantage to UML from a kernel development point of view is that it allows you to debug
the kernel using an ordinary source level debugger such as \command{gdb}. I will discuss how to
do this in more detail later in this document.

\subsection{Compiling UML}

The procedure for compiling the User Mode Linux kernel itself is simple. In what follows I will
assume you are using a 3.x kernel. The 3.x kernel comes with UML as one of the officially
supported architectures. First, unpack the kernel source code to a suitable working directory.
Next run the command

\begin{commands}
\$ make defconfig ARCH=um
\end{commands}

It is important to use the default configuration generated by \command{defconfig} as the
starting point for your kernel configuration. \emph{Do not try to use the configuration of the
  running kernel.} It is also important to add the \command{ARCH=um} option to the command line.
This tells the build system that you are cross compiling to a different architecture.

Next run either
\begin{commands}
\$ make menuconfig ARCH=um
\end{commands}
OR
\begin{commands}
\$ make gconfig ARCH=um
\end{commands}

It is important to consistently use the architecture specifier.

Under ``UML Specific Options'' be sure that ``Host filesystem'' is selected. This will allow the
UML kernel to access the file system of the host; an easy way to share files between the host
and a running UML system. Under the ``Kernel hacking'' option (on the top level menu) be sure
the ``Compile the kernel with debug info'' and the ``Compile the kernel with frame pointers''
options are both selected. These options make it possible to properly debug the UML kernel with
\command{gdb}. You may also want to set some other debugging related options as described in
Section~\ref{sec:configuring}. Save these changes.

You are now ready to build the kernel using the command
\begin{commands}
\$ make ARCH=um
\end{commands}

When the build is complete you will have an ordinary executable file named \command{linux} in
the root directory of the source tree. Before you can use it, you will need a root file system.

\subsection{UML Root File System}

User Mode Linux reads its root file system out of a file in the host's file system. This file
must be set up so that it contains all the normal programs and tools Linux needs to boot. You
may also want development tools or other programs inside your UML's root file system. Although
you may be able to download a root file system, matching the root file system with the precise
kernel version and options you want to use can be tricky. The reason for this is that during the
boot process, typical Linux configurations read various modules out of the root file system to
enable support for features needed by the startup scripts. These modules must be compatible with
the running kernel. If you are using a kernel version that is not compatible, the boot process
is likely to fail, or at the very least report a large number of errors.

Sometimes you can download a UML kernel along with a matching root file system. However, as a
kernel developer, the kernel you want to use is not arbitrary; it is a specially configured
kernel of your choosing. As a consequence of this, the ideal path is to build your own custom
root file system for kernel development. The procedure is as follows.

\todo{FINISH ME! The description below is very incomplete.}

\begin{enumerate}

\item Create a file to hold the root file system. This file may need to be rather large,
  depending on how much material you plan to install into the UML environment. Use a command
  such as
\begin{verbatim}
$ dd if=/dev/zero of=root_fs bs=1M \
    count=512
\end{verbatim}
  This command creates a file named \filename{root\_fs} with a size of 512 megabytes. This file
  is initially all zeros. It will be treated as raw disk image.

\item The image created above must then be formatted with a suitable file system.
\begin{verbatim}
$ mk2efs -j root_fs
\end{verbatim}
  This command creates an ext3 file system (which is the same as an ext2 file system with a
  journal created by the \texttt{-j} option) inside the \filename{root\_fs} file. You may need
  to specify the path to the \command{mk2efs} program; it is typically not in the path of
  ordinary users. The use of ext3, or some other journaled file system, is recommended. Since
  this environment will be used for kernel development, kernel crashes are likely and it is nice
  to have the extra safety inherent in using a journaled file system.

\item Mount the root file system so that you can access its contents. To do this you will need
  to be the root user. First create a suitable mount point. I suggest an empty directory named
  \filename{root}. Then issue
\begin{verbatim}
$ mount -o loop root_fs root
\end{verbatim}
  This command uses the loop back driver (which must be supported in your host kernel) to mount
  the file system contained in \filename{root\_fs} onto the mount point \filename{root}.

\item You will now find an empty partition beneath the mount point, ready for you to set up your
  root file system. Configuring a root file system for use with Linux is a non-trivial exercise.
  Lack of space in this document prevents me from going into the details here. Please refer to
  other documentation for more information.

\item Once the root file system is ready you should (as the root user) unmount it before you try
  to use it with UML.
\begin{verbatim}
$ umount root
\end{verbatim}
  You might want to make a backup copy, perhaps in compressed form, of your fresh root file
  system in case you destroy your working copy while setting up UML or doing kernel development.

\end{enumerate}

\subsection{Running UML}

Make sure the root file system is named \filename{root\_fs}. Execute \command{linux} to boot
User Mode Linux\footnote{You can name the root file system something else, in which case you
  need to add the \texttt{ubd0=} boot option to the command line to specify it.}. Log in as the
user root. Typically, depending on the root file system you are using, the password will either
be blank or also root.

Once you have logged in you can use the command
\begin{verbatim}
# mount none /mnt -t hostfs
\end{verbatim}
to mount the root of the host file system onto the \filename{/mnt} directory inside the UML
environment. This allows you to copy files to and from the host file system. You may now use
User Mode Linux in a manner very much the same as any other Linux system.

\todo{Talk about setting up disk partitions under UML.}

To debug the running UML process, open another window on the host machine. Use a command such as
\begin{verbatim}
$ ps aux | grep linux
\end{verbatim}
to search for information about the running UML process. You will find several entries because
UML is a multi-threaded application. Note the process ID of the first entry. Then launch the
\command{gdb} debugger, attaching \command{gdb} to the process of interest. For example
\begin{verbatim}
$ gdb linux 1234
\end{verbatim}
where 1234 is the process ID of the running UML system.

Once \command{gdb} has started you will want to execute the command \command{handle SIGUSR1
  nostop noprint}. I'm not entirely sure what this command does exactly but it has something to
do with the way \command{gdb} deals with multiple threads. In particular, without this command
\command{gdb} will stop the UML system frequently because of SIGUSR1 signals.

Notice that when \command{gdb} attaches to a process, that process is stopped. Your UML session
will appear dead. However, by issuing the \command{continue} command to \command{gdb} you can
cause your UML session to resume normally. Use \^C in the \command{gdb} window to interrupt the
UML session at any time. You can now set break points and single step the Linux kernel as you
might any other process.

\todo{Talk about how to set a break point inside a module.}

\section{Code Browsing Tools}
\label{sec:code-browsing}

The Linux kernel is very large and finding one's way around in it can be a major chore. To
simplify the navigation of large programs there exists a number of code browsing tools. I
recommend using one or more of these tools when working with the Linux kernel source. In this
section I talk about how to configure a few of these tools for use with the Linux kernel.

Note that you should only set up code browsing tools \emph{after} you've applied any patches to
the source code. Patches will, of course, modify files and change the line numbers were
functions are defined, etc. If you index the source and then apply patches you will want to
reindex the source afterward. However, it shouldn't matter if you've compiled the kernel first
or not. Code browsing tools are normally smart enough to ignore object files.

\subsection{ Cscope}

The \command{cscope} tool is a simple but effective code browser with a long history. It reads a
collection of C files and builds an indexed database that can be used to quickly look up
declarations and points-of-use for any symbol.

The script below launches \command{cscope} on the Linux kernel. You should edit the three
variable definitions at the top of the script to suite your system. The script does not index
the entire kernel code base. In particular it skips the driver hierarchy and only indexes the
x86 architecturally specific code. This makes the database size more manageable and reduces the
number of duplicate declarations the tool returns.

\begin{verbatim}
#!/bin/bash

# Set a few variables. Change here for your system.
CSCOPE_DIR=/home/peter/Linux/cscope
CSCOPE_FILE=$CSCOPE_DIR/cscope.files
LNX=/home/peter/Linux/linux-3.6.11

# If the database hasn't yet been created, then create it.
if [ ! -f $CSCOPE_FILE ]; then

  # Build file list. Exclude uninteresting regions.
  cd ~
  find $LNX \
    -path "$LNX/arch/*" ! -path "$LNX/arch/x86*" -prune -o \
    -path "$LNX/Documentation*" -prune -o \
    -path "$LNX/scripts*" -prune -o \
    -path "$LNX/tools*" -prune -o \
    -path "$LNX/drivers*" -prune -o \
    -name "*.[chxsS]" -print > $CSCOPE_FILE

  echo Creating database...
  cd $CSCOPE_DIR
  cscope -b -q -k

  echo Creating tags...
  ctags -L $CSCOPE_FILE

fi

# Run cscope
cd $CSCOPE_DIR
cscope -d
\end{verbatim}

In addition to using a dedicated code browsing tool, many programmer's text editors have a
feature that allows them to read a ``tags'' file containing cross reference information about
entities declared and defined in a program. The script above uses the \command{ctags} command to
create such a file for the \command{vim} editor.

It is natural to create a tags file for your editor at the same time as you create the
\command{cscope} database. This is because \command{cscope} will launch your editor
(\command{vim} is the default) whenever you ask it to display a file. Once in the editor it is
convenient to continue your browsing experience using editor tags commands.

If you are an \command{emacs} user instead, you can create a tags file for that editor with the
\command{etags} command. Set the EDITOR environment variable to \command{emacs} to override
\command{cscope}'s default editor setting.

\todo{FINISH ME! Need to talk about how to use \command{cscope} and the tags file.}

\end{document}
